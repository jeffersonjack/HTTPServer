\documentclass[a4paper]{article}

\usepackage{fullpage}
\usepackage{listings}
\pdfinfo{
  /Author (1561945)
  /Title  (HTTP Server Assignment - Documentation)
}

\title{HTTP Server Assignment \\
  \large Documentation}
\date{}
\author{}

\begin{document}

\maketitle

\section{Compilation and Usage}
The HTTP server is compiled using the {\bf make} command. The daemon can then
be started by typing {\bf ./httpd}.

\subsection{Selecting a Port}
If the daemon is run with no options (i.e. {\bf ./httpd}), then it will
attempt to bind to port number 80, and therefore superuser permissions are
required. A port number can be specified on the command line: {\bf ./httpd -p
  8080}. The port number should be between 1 and 65535, but note that only root
can bind to ports lower than 1024.

\subsection{Specifying a Directory}
If the server is started with root permissions, the server will serve files from
the {\bf /var/www/} directory. Otherwise, the current directory shall be
used. This default behaviour can be changed by using the {\bf -d} flag,
e.g. {\bf ./httpd -d /path/to/dir/}, but note that the daemon must have
sufficient permissions to access the files in that directory (see below).

\subsection{Users and Permissions}
For security reasons, if the daemon was run as the root user, it will bind to
the port and then attempt to drop its superuser permissions. The users {\bf
  httpd}, {\bf http}, and {\bf daemon} are tried (in that order). If none of
these users are found, or if switching to them fails for any other reason, the
HTTP server will fail to start.

If the daemon is not run as root, it shall continue to run as that particular
user.

If a custom directory is specified on the command line, the daemon will attempt
to change into that directory and proceed to serve files from there. The
directory must be accessible by the user under whom the daemon is running. For
example, if the daemon is started as root, and the {\bf httpd} user exists,
that user must have read permissions for files in the {\bf /var/www/}
directory.

\subsection{Success}
If the daemon is started correctly, no output will be displayed and a command
line prompt will be shown to the user. The daemon isn't attached to a tty, so
the terminal that was used to run the program can be closed safely.

\section{Code Structure}

\subsection{Setting Up}
The {\bf main} function (in {\bf main.c}) starts by checking that the
program was started correctly. If a port was specified (with the {\bf -p}
flag), type and range checks are performed. If the program was not started
correctly, for example if an invalid port was specified, or if an unexpected
option is encountered, the program prints a 'usage' message, and exits with an
error code. Once it is known that the program was started correctly, the process
is {\bf fork}ed, and the child process calls {\bf setsid()}. This creates
a new session group and sets the calling process as the group leader, thereby
detaching itself from the controlling tty. The {\bf getlisteningsocket()}
function is now called, to bind to, and listen on, the port. If that runs
successfully, a user ID check is performed and, if the process is running as
root, the process drops these permissions and runs as a different user (as
described above). The process then {\bf fork}s again, and the parent exits.

\subsection{Accepting Connections}
Now that the process is running as a daemon, and we have a listening socket, it
can start to wait for connections. This is done in the
{\bf acceptconnections()} function. This function runs in an endless loop,
each time calling {\bf accept()}, which, when a new connection is
established, returns the file descriptor of a client socket. A new thread is
then created (and detached - to avoid memory leaks) which handles the client's
request. The {\bf acceptconnections()} function then continues to wait for
new connections.

\subsection{Handling Connections}
When a connection has been established, the thread waits for, and reads, the
message from that client. An attempt is made to parse this message as an HTTP
request. If successful, an HTTP response message is generated. The URI from the
request is followed, resulting in one of the 3 following outcomes:

\begin{enumerate}
\item The URI points to a directory.
\item The URI points to a file.
\item The URI can't be followed.
\end{enumerate}

If the URI identifies a directory, a listing of that directory is generated, and
an HTML file is generated to display its contents. This is then sent back to the
user in the HTTP response.

If the URI identifies a file, the file is read and some HTTP response headers
(e.g. length, encoding, last modification time) are generated to describe the
file. These headers are then sent back to the client, followed by the file
itself.

If the URI can't be followed, the HTTP response code is set to 404, indicating
that the file cannot be found. A simple HTML message is then sent back to the
user in the body of the HTTP response.

Once the response headers and the response body have been sent back to the
client, the connection is closed and all memory that was allocated within the
thread is clan-ed up. The thread then exits.

\section{Limitations}
The HTTP request parser only recognises a message as a valid HTTP request if the
{\bf GET} method is used. With more time, this could have been extended to
include some of the other HTTP methods. However, as GET is the most useful, this
ins't a major issue.

\end{document}
